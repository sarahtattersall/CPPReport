\subsection{Testing}
  Unfortunately the down side to our modified TDD approach, which we did not really consider early on, was that in order to keep the build unbroken for everyone Sarah could not commit each test until someone was just about to implement it or had implemented it. This left her with a lot of uncommitted code which was hard to maintain. In the end she quite often ended up having to commit her tests which broke the build. Whilst it could be deemed ok to break the build on a planned basis, this meant that if someone else broke something it wasn't immediately clear as it should have been and so could sometimes go unmissed. Further problems arose when people had become competent with their delegated workloads; four people producing and modifying code at the same rate as Sarah writing her tests meant that she got rather behind on writing tests for new features and so quite often they were implemented before she'd written a test. This is something we'd overlooked when delegating responsibilities and meant that having a high code coverage quickly became an impossible task meaning our test suite has suffered significantly.

  The lack of code coverage combined with the constantly changing views on the client side, and the huge scope of the project, meant that quite often someone committed some code which broke another previously implemented feature (the fundamental thing tests are meant to catch!) After one stand up we weighed up as a group the time that it would take to increase our test coverage to an acceptable level and to fix many of the existing issues and had to decide to take the latter choice given the time frame of the project. In an ideal world we'd have performed a code freeze, where no new features could be added, and would have helped Sarah fix all the issues, writing good unit tests at the same time. Unfortunately time just got the better of us and we had to push on with functionality.

  In hindsight it would have been far better for Sarah to have read up on the frameworks, as planned, but to enforce everyone to write their own tests before writing a feature, which is the true TDD approach.Making use of her knowledge would most likely have got everyone up to speed quite quickly with testing and would have saved us all a significant amount of time debugging problems which were only found many days after they'd been committed, leaving the only thing left to do was a manual binary-search of the git log for the commit that broke the build. If we had instead broken a test when our code was fresh in our minds we'd most likely have fixed the problems a lot quicker too.