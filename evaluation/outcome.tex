\subsection{Usability}
	The user experience and interface are one of the most important factors in order to create an encapsulating and intuitive experience that users will enjoy. The user interface is designed to be as consistent and clean as possible, this was achieved with the help of twitter bootstrap to style elements of the interface.
		<<<Images?>>>
	The combination of rapid interface responsiveness (achieved via the use of backbone) with an intuitive user interface, makes using the system encapsulating and effortless. For example when a student edits their profile, they can quickly make changes to their key information inline, simply by clicking on what they want to change. The updates are confirmed via notifications that drop down at the top of the page, which are color coded to convey their message instantly. The simple yet empowering user experience is key to encouraging extended use of the system and provides a professional impression. Overall we are extremely pleased with the usability of the system, we have The simple yet empowering user experience that will encourage extended use of the system and provides a professional impression. 
\subsection{Performance}
	The performance in terms of the speed of page loads and responsiveness of the user interface is enhanced significantly via the use of Backbone. This reduces the number of server calls and increases client side processing to provide an instantaneous response to user interactions, creating a single page application type interface that supports the user experience and interface. We used New Relic in order to measure performance of the system in terms of user, application and server response times. It provides measures of page load times and can also provide user monitoring to give insight into how user interaction can be improved, for example by identifying poor performance patterns.
	One of the main factors influencing the performance of the system is error handling. If the system errors and breaks then this has a significant impact on the user experience, usability and performance. We have extensively tested the system using ... <<<<testing>>> ... and applied user acceptance testing in order to extensively test all parts of the system, and scenarios, to ensure that they perform as expected and to reduce the frequency of system errors to a minimum.
	Through the extensive testing, moitoring and the use of Backbone we have developed a fast and robust system thats high performance will support the slick user experience.
\subsection{Usefulness}
%TODO: DONT THINK THIS NEEDS A SUBSUBSECTION SINCE THERE'S NO OTHERS
\subsubsection{Evaluating Deliverables}
	The most significant outcome of the system lies in the overall improvement in finding, establishing and encouraging the relationship between students and companies. The system will make it easier for students and companies to get connected and makes the communication and application process more efficient.

\paragraph{Finding Information:}
	The use of Tags in the system provide an efficient way to link up different aspects of the system. Filters are also used to enable the user to quickly locate the information they require, filters can be applied as a text search over a given field or even in some cases enable the use of tags to show results with matching tags. This is particularly useful when companies are searching for students as they can search for students with specific skills, interests or by year group according to the tags that the student has added in their profile. Students can also search through placements and events using tags to suggest placements and events that match their tags. This will provide a more focused result when searching for information and ultimately will encourage increased connections as a result of the improved process of finding exactly what the user is looking for.

\paragraph{Communication:} 
	The system provides an integrated email facility that will Increase connection and enable more targeted emails via the use of tags, this reduces the amount of irrelevant information provided to students, enabling them to find the most relevant information to them and to sustain communications with companies.

\paragraph{Experience:}
	A dashboard where students can collate all their information, including their CV and other documents as well as key information, skills, interests, a bio and even a profile picture. This not only enhances the user experience for students but it also makes it easier for students to tailor their profile and be aware of exactly how the profile is viewed by the company. Having student profiles in this way also makes it easier and more efficient for companies to search for students and provides a more personal and professional environment than just a list of CVs and interests. The top three Events and Placements are displayed on the student profile, enabling students to remain in constant touch with the latest event and placement updates. This further enhances the connectivity between students and companies and makes it easier for student to find relevant information.

\paragraph{Companies:}
	One of the main improvements from the perspective of companies using the system is in searching for students. Previously companies would simply have to laboriously work their way through a list of imperial student cvs until they by chance found a student that they where looking for. In the new system companies can filter students by their skills, interests and year group and can quickly browse through student profiles that match the required criteria. They have the most important information clearly and consistently presented and can instantly access additional information such as cvs and other documentation. Once they have found a student that matches what they are looking for, they can contact the student at the touch of a button, contacting them via the integrated email system.
	Companies have a similar dashboard type profile that students have, enabling them to fill in key information, upload a company logo and create placements and events. This speeds the process and proves a more professional appearance of the program from the companies perspective, the company dashboard even provides analytical information such as the number of students that have viewed the page in the form of a line graph. 

\paragraph{Administration:}
	Due to the profile based system, the process of editing and updating company information, has now been delegated to the companies, this reduces the burden of the administration having to deal with all company information to each company updating their own information. This also increases the speed of the process because a company can instantly change their information at any time and send all that remains is for the changes to be approved by the administration.
	Creating and updating Placements and Events is also delegated out to individual companies which again disperses the amount of processing required by the administration, so that they only need to approve newly requested and pending changes to events and placements.
	Previously there was no way of telling how many students had even signed up to placements through the CPP program. The new system has a suite of analytical information available to the administration, this provides a breakdown key numerical summaries that are of interest to the administration for example the number of students that have viewed company profile pages (and even which students). This continued statistical monitoration will provide insight into the success or failure over certain years, enabling the CPP system to be fine tuned on a more factual grounding. 

\subsection{Overall Evaluation}
	As a group we believe that the combination of a clean and intuitive user interface combined with the array of empowering new features has created an encapsulating, user profile oriented experience that established, encourages and supports the relationship between students and companies.
	The only aspect that we feel could have been incoperated further to enhance the overall design process would have been to increase user testing however we have made continuous use of the feedback from our supervisors, which gave us points of improvements for functionality and interface design.
	We believe that the most significant achievement is the fact that the project will inevitably enhance student and company relations throughout all stages of the Corporate Partnership Program.
