\subsection{Usefulness}
%TODO: DONT THINK THIS NEEDS A SUBSUBSECTION SINCE THERE'S NO OTHERS
\subsubsection{Evaluating Deliverables}
  The most significant outcome of the system lies in the overall improvement in finding, establishing and encouraging the relationship between students and companies. The system also enhances a number of different aspects of the Coperate Partnership Programme.

\paragraph{Finding Information:}
  The use of Tags provide an efficient way to link up different aspects of the system. Filters are also used to enable the user to quickly locate the information they require, filters can be applied as a text search over a given field or in some cases enable the use of tags to show results with matching tags. This is particularly useful when companies are searching for students as they can search for students with specific skills, interests or by year group according to the tags that the student has added in their profile. Students can also search through placements and events using tags to view only placements and events that match their tags. This will provide a more focused result when searching for information and ultimately will encourage increased connections as a result of the improved process of finding exactly what the user is looking for.

\paragraph{Communication:}
  The system provides an integrated email facility that will increase connection and enable more targeted emails via the use of tags, this reduces the amount of irrelevant information provided to students via email, enabling them to find the most relevant information to them and to sustain immediate communications with companies.

\paragraph{Experience:}
  ***LAYOUT We feel that one of the main deliverables of our project is the improved user experience. This impacts Students, Companies and Administration in the following ways:

 \paragraph{Student:} 
 	A key aspect of improving the student user experience is created via the student dashbaord. This improves the student experience as they can collate all their information, including their CV and other documents as well as key information, skills, interests, a bio and even a profile picture. This makes it easier for students to tailor their profile and be aware of exactly how the profile is viewed by the company. The top three Events and Placements are displayed on the student profile, enabling students to remain in constant touch with the latest event and placement updates. This further enhances the connectivity between students and companies and makes it easier for student to find relevant information.

\paragraph{Companies:}
  A major improvement from the perspective of companies using the system is in searching for students. Previously companies would simply have to laboriously work their way through a list of imperial student cvs until they by chance found a student that they were looking for. In the new system companies can filter students by their skills, interests and year group and can quickly browse through student profiles that match the required criteria. They have the most important information clearly and consistently presented and can instantly access additional information such as cvs and other documentation. Once they have found a student that matches what they are looking for, they can contact the student at the touch of a button, contacting them via the integrated email system.

  Companies have a similar dashboard type profile as students, enabling them to fill in key information, upload a company logo and create placements and events. This speeds the process of filling out company infromation that they would previously have to submit via email and proves a more professional appearance of the program from the companies perspective. The company dashboard even provides analytical information such as the number of students that have viewed the page. This will also increase the interaction from companies and therefore lead to increased student relationships. 

\paragraph{Administration:}
  Due to the profile based system, the process of editing and updating company information, has now been delegated to the companies, this reduces the burden of the administration having to deal with all company information. This also increases the speed of the process because a company can instantly change their information at any time, all that remains is for the changes to be approved by the administration.

  Creating and updating Placements and Events is also delegated out to individual companies which again disperses the amount of processing required by the administration, so that they only need to approve newly requested and pending changes to events and placements.

  The new system has a wide range of analytical information available to the administration, this provides the possibility of producing a breakdown of key numerical summaries that are of interest to the administration for example the number of students that have viewed company profile pages (and even which students). Continued statistical monitoration will provide insight into the success or failure over certain years, enabling the CPP system to be fine tuned on a more factual grounding. 