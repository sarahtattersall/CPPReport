\section{Conclusion}
As the size and success of the Corperate Project Program continues to increase, the requrement for a new system to handle the wide range of different parts of the system is becoming increasingly eminant. CPP 2.0 Connected will lead to a complete renovation in the way that the current system functions, it will change the way users interact with the system at all levels and lead to enhanced connectivity and increased automation.
	
	\subsection{Main Challenges}
		Overall as a group we have worked together effectively, developing an open and supportive development environment to encourage development, supported by an underlying Agile design methodology. The only aspect that could have been incoperated further to enhance the overall design process would have been to increase user testing, the main of the current design feedback has been through our supervisors which gave us points of improvements for functionality and interface design.
		The restricted time allocated for the project in association with the decision to use some new technologies posed a large challenge in terms of ensurung that we remained on track and implemented the key features required, we think that as a result of continuous monitoring and planning, through frequent standup and supervisor meetings, we managed to guide our project to provide a powerful and user oriented system to improve all aspects of the corperate partenership program. 
	
	\subsection{Future Improvements}
		Due to restricted time constraints of the project, we concentrated on ensuring that the main requirements where fulfilled in the project. As a whole we found that the majority of features that we set out to implement where successfully acheived however there are some additional feautures that given more time we would like to incoperate to the project. 
		One of the additional features that we would like to include would be a glassdoor type internal system, this would enable student to review placement, giving experience aswell as tips and inside information to provide students with insight into first hand experience of placements, interviews and application processes. Over time would build up and internal repertoire of student experience to provide students with the best preparation.
		We also thought that translating the project into an API would open it up to <<<<API things>>>
		Maps gmap
		More restriction on validation

	\subsection{Team Success}
		Overall as a group we have worked together effectively, developing an open and supportive development environment to encourage fast progression, supported by an underlying Agile design methodology.
		Investing in the use of new technologies presented obvious risks to the development process, however we feel that we have made considered choices into the technologies available to us and have used them to achieve results that would have otherwise not been feasible.
		The final system coalesces the different aspects seamlessly, which is integrated with the intuitive and innovative interface to enable the user to effortlessly achieve tasks.
		From the division of the team into different specialist areas, the experience gained by individual members varies. Each team member has gained new exposure to Ruby on Rails, Backbone and Coffeescript throughout the project, as well as previously used HTML and CSS experience. Within the specialist groups we found that we became proficient using the technologies the where most relevant for example front-end programmers became proficient in the use of backbone and similary back-end developers becaome proficient using Ruby on Rails. Due to the open and supportive development environmnt we found that overall we all became accustomised to the new technologies used, leverging the advantages they offered to compensate for the time invested in learning them. 
		The team also learnt the benefits of incorperating open source libraries as gems to enable faster development and became familiar with different Agile development techniques. We all feel that our ability to communicate and collaborate as developers has been greatly imrpoved through the project.
		
		
