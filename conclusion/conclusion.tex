\section{Conclusion}
As the size and success of the Corporate Partnership Program continues to increase, the requirement for a new system to handle the wide range of different parts of the system is becoming increasingly eminent. CPP: Connect Connected will lead to a complete renovation in the way that the current system functions, it will change the way users interact with the system at all levels and lead to enhanced connectivity and increased automation.
	
	\subsection{Future Improvements}
		Due to restricted time constraints of the project, we concentrated on ensuring that the main requirements were fulfilled in the project. As a whole we found that the majority of features that we set out to implement were successfully achieved however there are some additional features that given more time we would like to interoperate to the project.

		One of the additional features that we would like to include would be a glassdoor type internal system, this would enable student to review placement, giving experience as well as tips and inside information to provide students with insight into first hand experience of placements, interviews and application processes. Over time would build up an internal repertoire of student experience to provide students with the best preparation.
		% TODO PETE API
		*** We also thought that translating the project into an API would open it up to ...
		
		An enhancement feature that we could include to events and placements would be to include a Google map view of the location, preventing the student from having to look up where an even or placement is. This would need to interface with the google maps API and would be an enhancement that we would like to include given additional time.

	\subsection{Team Achievements}
		Overall as a group we have learnt to work together effectively, developing an open and supportive environment to encourage fast progression, supported by an underlying Agile design methodology.
		Investing in the use of new technologies presented obvious risks to the development process, however we feel that we have made considered choices into the technologies available to us and have used them to achieve results that would have otherwise not been feasible. The final system coalesces the different aspects seamlessly, which is integrated with the intuitive and innovative interface to enable the user to effortlessly achieve tasks.
		
		As a results of the division of the team into different specialist areas, the experience gained by individual members varies. Each team member has gained new exposure to Ruby on Rails, Backbone and CoffeeScript throughout the project, as well as previously used HTML and CSS experience. Within the specialist groups we found that we became proficient using the technologies that were most relevant for example front-end programmers became proficient in the use of Backbone and similarly back-end in Ruby on Rails. Due to the open and supportive development environment we found that we all became accustomed to all the new technologies used, leveraging the advantages they offered to compensate for the time invested in learning them.

		The restricted time allocated for the project in association with the decision to use some new technologies posed a large challenge on the project in terms of development time. To counteract this each member needed to learn to utilise the new languages fully, including using programming sites such as code school to build specifically on the coding experience we have already obtained throughout our degree. We also appreciated the use of gems to enable faster development by incorporating existing libraries into our project. 

		In terms of team management the team became familiar with different Agile development techniques. We all feel that our ability to communicate and collaborate as developers has been greatly improved through the project, we have learnt the benefits of having team meetings and have gained experience in expressing our ideas and communicating within meetings. As a team we have also learnt the importance of staying on track through weekly iterations in order to deliver on time. Finally we have expanded on our experience in using technologies which support collaborative development such as GitHub and collaborating remotely via Skype calls.
