\subsection{Challenges}
As well as design challenges we faced many technical challenges along the way with the most significant being detailed below.
  \subsubsection{New technologies}
    Our group faced varying levels of web development abilities when coming into this project; Pete and Tom were both experienced with releasing professional websites for companies, where as Tom, Jack, and Sarah had only done a small amount of web development before. [CORRECT ME IF I'M WRONG]
    With the exception of Peter, every single one of us was new to Ruby on Rails before this project. Everyone had used JavaScript to a varying degree, but only Peter and Sarah had used CoffeeScript before, and everyone was new to Backbone.js.

    This meant that there was clearly a steep learning curve for our project and that if we were all to learn every single new aspect in detail we would spend more time learning than producing a professional product.
    We therefore decided to deligate responsibilities into three groups:
    \begin{itemize}
      \item \textbf{Front-end development -} Here Tom, Tom and Jack researched the client-side technologies we needed to know and focused on front end implementation and issues on Github.
      \item \textbf{Back-end development -} Since Pete was already familiar with Ruby on Rails he felt relatively confident to tackle the backend development of our project. Once implemented he could then join in with the client side development with a fast-track introduction from the others. As Sarah is more interested in back end projects she helped Pete with this in order to pick up some knowledge for RSpec testing.
      \item \textbf{Testing -} Sarah had done quite a lot of indetail testing in her summer internship with Amazon and so decided she would like to become proficient with RSpec and Jasminerice for the group. In order to do this she observed and worked on both server side and client side issues to help with the knowledge of the languages.
    \end{itemize}

    We feel that this solution helped to tackle the large amount of work that needed to be done in a relatively short time frame.
  \subsubsection{Updating our Gems}
    Leading on from the previous point that we had a lot of new technologies to learn, this meant that we did not always understand the complete inside and outs of them when using them, just enough knowledge to procede with our project. This meant that in December when we decided to update all of our gems and some features unexpectidly broke we were left quite confused. As our test suite was not complete (documented later), they managed to go unnoticed for quite a while.
    One such feature that was observed was saving Backbone models from client to server the server returned OK, however the error call back was called instead of success. This was first noticed when adding new tags to a students skill set, the tag would only be added on refresh (since it saved on the server).
    We were absolutely sure that it must have been something we'd done to the code, that slipped past testing sicne consulting the Backbone.js documentation was no obvious help. Eventually we tracked the issue down to the Gem update commit that convinced us it must have actually been to do with updating them Gems.
    This issue took a very long time to fix and involved many searches, a Stack Overflow question, and was only fixed by consulting the source code, the real problem is documented in the appendix.
    We believe these issues came fro,m
  \subsubsection{Think of something else}