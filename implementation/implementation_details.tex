\subsection{Implementation Details}

	\subsubsection{Evaluating Technologies}
		In the initial stages of the project we decided to evaluate the different available technologies that we could use to create CPP: Connect. We considered carefully the advantages and disadvantages of different technologies when applied to our project and came to a group conclusion on the technology choices.

	\subsubsection{Client Side Technologies}
		Due to the user experience oriented nature of our project we wanted to provide a highly responsive user interface. 

		In order to achieve a rich and highly interactive user experience we ideally wanted to create a single page application type interface that provides rapid response and updates to the interface. We therefore researched into JavaScript libraries such as Backbone and Knockout which provide means of producing data-rich web applications.  
		
		Backbone provides a Model View Controller (MVC) pattern which improves the organisation and structure of client side code, enhancing code production and maintenance. The Backbone MVC design pattern also leads to separation of concerns as it de-couples data from presentation logic. Backbone integrates seamlessly with Rails via the use of the gem backbone-on-rails and reduces the volume of calls made to the rails server. This is because HTML content is served only once on the first access of the site and consequent infrequent server calls are only used to transfer data in a lightweight JSON format. Reducing load on the server and enabling more client processing results in a faster interface that feels more responsive and provides an enhanced user experience.
		
		When deciding to incorporating Backbone we had to compare the cost in terms of development progress as a result of members having to learn a new framework against the enhancement to the user interface that would result. We concluded that we needed to incorporate Backbone because it would be impossible to achieve an interface as responsive and fluid without using Backbone.

	\subsubsection{Server Side Technologies}
		Deciding which server-side technology to use involved considering the storage and security requirements listed above as well as group experience and support for the required functionality of the project. Ruby on Rails provides a succinct solution to many of the key aspects that we required from our application. Ruby on Rails also couples well with Backbone, and although there may be some duplication in mirroring the MVC structure in the back-end this is outweighed by the benefits of their integration. There is also a vast amount of support for using Backbone and Ruby on Rails together as they are frequently used to develop web applications including Gems and tools such as scaffolding to setup key aspects automatically and the ability to minify code to decrease site loading time.
		
		One main factor in our decision to use Ruby on Rails was the availability of open source libraries (``Gems'') available this would simplify the code of many features instead of reinventing the wheel.
		There were several members of the group that had experience in PHP and alternative sql databases, one of the group members had prior experience in using both Ruby on Rails and alternatives and preferred using Ruby on Rails due to its elegant syntax. We decided that due to its concise syntax, integration with Backbone and Gem availability, with the support of the one group member that was confident in using Ruby on Rails we could learn to use, and gain from the advantages of using, Ruby on Rails.
