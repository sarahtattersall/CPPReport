\subsection{Technical Overview}
	The following requirements need to be supported by the technologies chosen to implement CPP 2.0:
		\begin{itemize}
		  \item \textbf{User Interface:} Produce an innovative user experience providing a responsive profile based experience that encapsulates the user and makes the system enjoyable and intuitive to use. This results in the need for a fast client side interaction with a professional user interface.
		  \item \textbf{Database:} User, Company and Admin information needs to be stored in an organised and easily accessible format.
		  \item \textbf{Security:} Handling student private data such as their CVs is sensitive information, we need a way to securely store such information 
		\end{itemize} 

	\subsubsection{Server-Side Overview}
		CPP 2.0: Connected uses
		\paragraph{Ruby on Rails\cite{ror}} as a web framework for storing persistent data and communicating it to the front-end of the system. Ruby on Rails uses a Model View Controller design pattern to organise the back-end structure, providing secure and efficient storage of persistent user information. The following schema entity relationship diagram details the setup of the different entities within the back-end Ruby on Rails server.
		<ERD>
		Ruby on Rails provides an elegant and succinct way to communicate with the front-end implemented in Backbone, as well as enabling  

	\subsubsection{Client-Side Overview}
		CPP 2.0: Connected uses:
		\paragraph{Backbone\cite{backbone}} to provide a highly interactive and responsive client-side experience. The Backbone library is essentially broken down into Models, Collections and Views that connect to the Rails server API using a RESTful JSON interface. Models contain the attributes of the data entities as well as the logic surrounding them, Collections are ordered sets of models with a rich API of enumerable functions and Views concern the presentation logic with declarative event handling. Backbone implements a Model View Controller (MVC) design pattern which applies the ‘separation of  concerns’ principle, this results in improved structure and organization of code.
		\paragraph{CoffeeScript\cite{coffeescript}} is used as an alternative to Javascript due to its layer of beautiful syntactic sugar that allows us to produce Javascript in a really succinct and efficient way. This results in more efficient, neater code as well as longterm increased code production speed.
		\paragraph{Twitter Bootstrap\cite{bootstrap}} is a front-end framework used to present a clean and simple user interface that supports dynamic scaling to different screen sizes. We chose Bootstrap for this reason and because it provides many useful JavaScript plugins such as the typeahead for our student degree and tag input fields. It provides nice visuals with very little setup which was perfect for our timeframe of this project. We would not have had time to have designed the CSS ourselves.